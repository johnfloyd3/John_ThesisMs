\section{Introduction}
Interesting direct manifestations of helium electronic correlations have been found in certain highly asymmetric doubly excited states which are associated with highly asymmetric classical configurations. The most interesting one is the frozen planet configuration (FPC), that is characterized by having a stable dynamics against autoionization. These regions can support quantum states called frozen planet states, that can transform under external electromagnetic periodic driving in quantum states that propagate along the classical trajectory without dispersion. Previous studies suggest that when a static electric field is added to the periodic driving an improvement on the stablization of the nondispersive wave packets can be achieved, this is a consequence of the effect this perturbation has on helium classical dynamics, confining its motion to a vicinity of the field polarization axis.

In this project, we will propose a method to investigate highly doubly excited states of helium under periodic driving and an external static electric field. The eigenvalues calculated from the Schr\"odinger equation for this system allow us to analize the stabilization properties of the states, such as their energies and their decay rates, from which the lifetime of the resonance states can be computed. The localization of the states can be analized using electronic projection of phase space, Husimi's densities. The latter properties will be tested as function of the static electric field.

This research work is intended to investigate the effect of a static electric field perturbation on frozen planet states and to analize their corresponding localization properties and lifetimes. Another pretention of this research project is to develop a similar analysis on nondispersive wave states, which are special states that emerge in the Floquet spectrum of the Hamiltonian, distinguished by their localization properties and by the fact that they follow concrete classical trajectories. Thus, Husimi's densities and stabilization properties for these wave packets will be studied in an external static perturbation.

\section{Background}

During the last decade, new techniques were created to develop coherent light sources that permitted to create ultra short pulses \cite{doi:10.1021/acs.chemrev.6b00453}, down to a few tens of attoseconds ($ 1\, \textrm{as}=10^{-18} \, \textrm{s} $), in the extreme ultraviolet spectral region. These pulses allowed us to access a very fast dynamics, such as electron motion induced by light-matter interactions. Among many of the vast range of applications this impressive technologies has brought, one can mention the observation of the valence electron motion \cite{article}, the caracterization of electron wave packets in helium, monitoring the birth of autoionizing resonances \cite{Gruson734} or the decay of a core vacancy \cite{article2}. In \cite{Schultze1658}, it was shown that the photoionization delay, this is,  the time for the photoelectron emission \cite{cavalieri2007attosecond}, depends on the excitation energy and the underlying ionic core structure in the attosecond time scale. On the other side the capacity to control electron dynamics is very important for some areas in chemistry, the reason for this is that time dependent electronic density is responsible for bond formation and bond breaking in molecules, this new research field is sometimes called attosecond chemistry \cite{RevModPhys.81.163,article3}. The full understanding of these experimental results requires background theoretical interpretations that can explain the intrinsic phenomenology and also be able to perform predictions. The interaction of light with the helium atom seems a promising way to follow this path, since it is the simplest example of a multielectron atom, but its dynamics is nonintegrable from both, classically and quantum mechanically \cite{thesisalejo}.

The helium atom is a special case of a three body problem with a Coulomb interaction between the two electrons. The classical dynamics of this system is non-integrable, then its phase space is characterized by being chaotic in general \cite{richter1993intra}, with only small regions of regular motion. An important consequence of the loss of integrability is the failure of first quantization attempts on the basis of Niels Bohr's postulates \cite{doi:10.1080/14786441308634955,einstein1917quantensatz,doi:10.1080/14786441308634993}. The correspondence between the chaotic-regular classical dynamics and the non-integrability of the quantum system \cite{RevModPhys.72.497} could be understood after the development of modern semiclassical theory in the second half of the 20th century \cite{gutzwiller1967phase,gutzwiller1971periodic} and the subsequent semiclassical quantization of helium \cite{ezra1991semiclassical,leopold1980semiclassical}.

Early observations made by Madden and Codling \cite{PhysRevLett.10.516} demonstrated that two-electrons doubly excited states were highly correlated, and they cannot be described by a model based on independent particle quantum numbers, therefore doubly and highly excited states of helium are of great interest among experimentalists and theoreticians. 

Strong electronic correlation have been found, specifically, in highly asymetrically excited states of unperturbed helium which are well localized around the classical frozen planet configuration \cite{richter1990stable,richter1992novel}. These states transform under near-resonantly periodic driving into nondispersive wave packets (NDWP), which are quantum objects that allow the possibility to localize the electronic population in phase space \cite{weinacht1999controlling,chen1998observation}, stablishing therefore a connection between a quantum wave function and its corresponding classical system \cite{schrodinger1926stetige}. The existence of two-electron NDWP was theoretically proven for one dimensional \cite{BUCHLEITNER2002409,schlagheck1999nondispersive}, planar \cite{thesisjavier,Madronero,PhysRevA.101.013414}, and recently for the full three dimensional helium atom \cite{thesisalejo}.     

However, the localization properties of NDWP strongly depend on the stability of the dynamics of electron in the FPC, this is, the time that the configurations take before the atom ionization. This stabilization is rapidly broken when a small deviation of collinearity is present \cite{Schlagheck2003}. As it was proven  in \cite{schlagheck1998classical} a static electric field applied in the direction of the driving polarization axis can be used to enforce the stabilization of the driven classical configuration in all three spatial dimensions.

The nondispersive wave packets dynamics was already explained for the collinear helium case in \cite{schlagheck1998classical}. There, it was shown that applying a static electric field with a strength of up to 20 percent of the driving field amplitude, it was possible to modify the ionization rates in an apprecciable manner. Certain specific values for the static perturbation allow to modify the wavepacket's lifetime significantly, for example, from $ 1.8 \times 10^{6} $ to $ 1.0 \times  10^{4} $ field cycles . Nevertheless, ionization rates  don't always decrease with increasing static field, instead, an erratic dependence between ionization rates and applied dc fields has been found, this behavior has not been explained yet.

It is important to realize the fact that even though bound states of any atom become quasi bound under the influence of a dc field, we are only working on the region of highly doubly excited states. Thus, on the basis of the previous cited phenomenona present on the classical FPC, we try to find appropiate values of the static field, such that ionization rates and other localizations properties get substantially modified.

The purpose of this research work is to investigate the effect of a static electric field perturbation on frozen planet states and to analize their localization properties and lifetimes. Thus we can exhibit the propiate values for the static field to get the best lifetimes of the resonances. These same properties will be determined for NDWP in the full three dimensional case in presence of the static perturbation.