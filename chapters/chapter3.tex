\section{The classical frozen planet configuration of Helium}

The classical dynamics of the helium atom under the influence periodic and static electromagnetic perturbation is described with the Hamiltonian

\begin{equation}
	H=\frac{\vec{P}_{1}^{2}}{2}+\frac{\vec{P}_{2}^{2}}{2}-\frac{2}{\abs{\vec{r}_{1}}}-\frac{2}{\abs{\vec{r}_{2}}}+\frac{1}{\abs{\vec{r}_{1}-\vec{r}_{2}}}+(F \cos \omega t +F_{\textrm{st}})(\textbf{e}_{x}).(\vec{r}_{1}+\vec{r}_{2}), 
\end{equation}
where $ \vec{r}_{i}=(x_{i},y_{i},z_{i}) $ and $ \vec{P}_{i}=(p_{ix},p_{iy},p_{iz}) $ are the position and momentum of particle $ i=1,2 $ respectively. Besides $ F \cos\omega t $ is a linearly polarized driving at frequency $ \omega $ and amplitude $ F $ ($ \textbf{e}_{x} $ represents the unit vector along the x-axis). Besides there is an additional static field strength $ F_{\rm st} $. The classical dynamics goberned by this Hamiltonian is invariant under the scaling transformations

\begin{equation}
	\vec{r}_{i} \rightarrow  N \vec{r_{i}} \hspace{0.5cm} (i=1,2),
\end{equation}
\begin{equation}
\vec{p}_{i} \rightarrow  N^{-2} \vec{p_{i}} \hspace{0.5cm} (i=1,2),
\end{equation}
\begin{equation}
t \rightarrow  N^{3} t,
\end{equation}
\begin{equation}
\omega \rightarrow  N^{-3} \omega,
\end{equation}
\begin{equation}
F \rightarrow  N^{-4} F,
\end{equation}
\begin{equation}
F_{st} \rightarrow  N^{-4} F_{st},
\end{equation}
\begin{equation}
H \rightarrow  N^{-2} H,
\end{equation}
where $ N $ is an arbitrary, real positive number. Due to this invariance the intrinsic quantities depend only on the action integral over one cycle of the Kepler oscillation of the inner electron, where $ x_{2} $ and $ p_{2} $ are its position and momentum, respectively.

Un this work we are interested in a particular solution of the classical equations of motion, which is called the frozen planet configuration (FPC) \cite{RevModPhys.72.497,PhysRevLett.65.1965}. This is an asymmetric configuration where both electrons are located on the same side of the nucleus. While the inner electron oscillates rapidly in extremely eccentric Kepler trajectories around the nucleus, the outer electron remains nearly frozen around some equilibrium distance. The importances of this configuration lies in the fact that it is classical stable under autoionization and defines a relatively large region of regular motion in the chaotic unperturbed phase space of helium.

The regularity of the classical FPC is visualized in a Poincaré surface section. The necesity of a method like this stems from the fact that the driven helium dynamics takes place in a five dimensional phase-space, spanned by the positions and momenta of the electrons, therefore a reduction of the dimensionality is a practical choice to be made. Here, each point $ (x_{1},p_{1} ) $ represents the position and momenta of the outer electron when the inner electron collides with the nucleus. In the case of collinear the phase-space contains a large region of bounded motion, as can be seen in figure.

When the driving field is switched on, the phase space turns mixed regular-chaotic. Now, the phase space exhibits an intrinsic regular region centered around the equilibrium position of the configuration \cite{Schlagheck2003}, which comes from the interelectronic perturbation, besides there is an additional regular island immersed in the chaotic sea which is a consequence of the non linear resonances between the external driving and the unperturbed oscillation of the unperturbed oscillation of the outer electron. When the phase of the periodic driving is varied as time increases, the intrinsic island remains at rest, while the field-induced resonance island oscillates around the intrinsic island with the same period of the driving field.  

There is a notable difference in the time scales for both electrons oscillation modes, being the Keppler oscillations of the inner electron almost 15 times faster than the slow oscillation of the outer electron around its equilibrium distance. The separation of time scales allows one to treat the outer electron dynamics in the framework of adiabatic approximations, and,  consequently, to define an effective potential to describe its slow dynamics. From the shape of this potential, it is possible to obtain the natural scale for the field strength, which indicates the maximum field that can be applied to the configuration without ionizing it, and the frequency scale, that is given bye the curvature of the potential at its minimun. These scaled quantities in atomic units are given by

\begin{equation}
x_{\textrm{min}}= 2.6 S^{2},
\end{equation}
\begin{equation}
\omega_{I}= 0.3 S^{-3},
\end{equation}
\begin{equation}
F_{I}= 0.3 S^{-4},
\end{equation}
where 

\begin{equation}	
	S=\frac{1}{2\pi} \int p_{2}dx_{2}.
\end{equation}
In a quantum description the action variable is replaced by the principal number of the inner electron, thus the scale quantitites adopt the form of 

\begin{equation}
x_{\textrm{min}}= 2.6 N^{2},
\end{equation}
\begin{equation}
\omega_{I}= 0.3 N^{-3},
\end{equation}
\begin{equation}
F_{I}= 0.3 N^{-4}.
\end{equation}

In this description of the problem, there are eigenstates of the spectrum that are well localized along the frozen planet orbit, this is due to the large phase space volume occupied by the stability region of the FPC . These states are the \textit{frozen planet states} (FPS), whose existence has been demonstrated on 1D \cite{Schlagheck2003}, 2D \cite{Madronero}  and 3D helium  \cite{Richter_1992}. FPS can be recognized by proper analysis of their spectral properties sucha as decay rates and the expectation value of $ \cos\theta_{12} $. Since these states are localized along a collinear configuration the expectation value of $ \cos\theta_{12} $ is expected to be approximately one. Besides, they ought to have long lifetime compared to other resonances in the same energy region which is the small decay rate condition. Nevertheless, none of these features allows to completely caractize FPS, then the localization properties of electronic density in configuration and phase space have to be considered.

\subsection{Identification of frozen planet states}

Frozen planet states are eigenstates in the helium spectrum that are localized in the stability region of the FPC. Hence, given a collinear configuration one main feature for a state to be considered a FPS is that the expectation value of the cosine of the angle $ \theta_{12} $ between the radius vector $ \vec{r_{1}} $ and $ \vec{r_{2}} $ has to be close to unity. Moreover, since these resonance are expected to posses a large stability against autoionization, long lifetimes, or, equivalently, small decay rates are needed. Nevertheless, these two requirements are not sufficient to guarantee the FPS nature of a given state. Instead, frozen planet states can be identificated unambiguously by their localization properties in configuration and phase space. Nevertheless, this can´t be performed directly due to the dimension of the spaces involved (e.g six for configuration space). Then, visualization requires projection of the probability density. Thus, two projections of the electronic densities are used in configuration space, these are, conditional probability distributions for $ \theta_{12} $ or one-electron probability densities, whereas, for phase space projection we use Husimi densities.

\section{Frozen planet states in N=6 helium spectrum}

In this section, we present the principal characteristics of frozen planet states lying below the $ N=6 $ ionization threshold. These states are of major interest, as we will see later, in the determination and characterization of non dispersive wave packets.

In order to characterize the FPS we first have to determine the section of the spectrum corresponding to states converging to $ N=6 $ ionization threshold. To find these states (figure), we diagonilize the hamilotinan in (), where we adjust the parameter, to find only states in the energy region of interest. FPS are characterized by their expectation value and their low decay rates. Figure () shows the resonance lying above $ N=6 $ ionization threshold,  with the potential FPS identified with red circles. It is seen that these states match the necesary conditions. Energies, decay rates and cos for the first members of these series are shown in table.

To completely categorize these states as FPS we still have observe their localization properties. In figure () the conditional probability densities for the first two states in the series are shown. For stae It is observed that the maximum probabily for the inner electron is close to the necleus at , whereas for the outer electron, labeled bye , the maximum probability is at .

On the other hand Figure 3.6 displays the Husimi distributions of the first four 1 S FPS converging to the 6th series for planar and three-dimensional helium. Comparing these plots with the classical configuration, we observe that the maximum of the probability density of the ground state is localized at the equilibrium position of the outer electron in the classical FPC given by Eq. (3.2). On the other hand, excited FPS are localized along periodic orbits with higher energy of the classical configuration.

