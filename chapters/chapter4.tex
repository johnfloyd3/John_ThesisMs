

Usually wave packets spread as time evolves. Nevertheless, it has been shown that, in some quantum systems such as one-electron Rydberg systems, this spreading could be overcome with non linear resonancses between the system and an external periodic driving. This procedure allow to create Non dispersive wave packets (NDWP), which are well localized and follow a classical periodic orbit without spreading. The dynamics of frozen planet states gives rise to wave packet that propagate along Keppler trajectories of arbitrry eccentricity and whose lifetime can be considerably long compared to any other state in the spectrum.

In section () we will review generalities of the frozen planet configuration under periodic driving. Then, section 2 reviews the characterization of non dispersive wave packets for helium. In the last section of this chapter, we shall review how an aditional static field perturbation can modify the fundamental properties of nondispersive wave packets

\section{Classical frozen planet configuration under periodic driving}

Now we consider the FPC in its driven case, this is, when the field frequency is chosen near resonant with the frequency of the periodic orbit.

The evolution of this system takes place in a five dimensional phase space spanned by  the positions and momenta of the electron and the $ \omega t $ of the driving field. Therefore, a complete visualization is not possible. However, for $ w $ and $  $, the separation between the rapid Keppler oscillations of the inner electron and the slow oscillation of the outer electron, makes possible to map the phase space structure onto a two dimensiones surface, by a two-step section Poincaré method. In the first step, the position and momenta $ (x_{1},p_{1}) $ of the outer electron are plotted every time the inner electron reaches the nucleus, this eliminates the fast Kepler oscillations. Then, these points are connected by cubic interpolation to yield a continues trajectory that is used to perform a second Poincaré section by fixing the phase of the driving field $ \omega t=\phi_{0} $.

Figure () shows the two step Poincaré section obtained for fixed field frequency $ \omega = 0.3N^{3} $a.u., and field amplitude $ F=0.005 N^{-3} $a.u., and field phases $ \omega t=0 $, $ \omega t=\frac{\pi}{2} $ and $ \omega t= \pi $.